\documentclass[conference]{IEEEtran}
\usepackage{hyperref}
\usepackage{tabularx}
\usepackage{cite}
\usepackage{graphicx}
\usepackage{algorithm}
\usepackage{listings}
\usepackage{microtype}
\begin{document}
\title{Towards a faster web\\high-performance computing on the browser}

\author{\IEEEauthorblockN{Matti Jokitulppo}
\IEEEauthorblockA{Aalto University\\
School of Science\\
Email: matti.jokitulppo@aalto.fi}
\and
\IEEEauthorblockN{Olavi Haapala}
\IEEEauthorblockA{Aalto University\\
School of Science\\
Email: olavi.haapala@aalto.fi}
\IEEEauthorblockN{Mehrad Mohammadi}
\IEEEauthorblockA{Aalto University\\
School of Science\\
Email: mehrad.mohammadi@aalto.fi}
}

% make the title area
\maketitle

\begin{abstract}
The abstract goes here.
\end{abstract}

\section{Introduction}
Until now, programmers have had little option but to use Javascript, or
languages that compile down to it like Elm or PureScript when writing modern
browser web applications applications. Furthermore, With the ever-increasing
advent of mobile computing devices and hybrid mobile applications, fast
browser-side code is more topical now than ever. However, the high-level,
interpreted nature of Javascript makes performance difficult to achieve.
Besides speed, modern Javascript frameworks and languages that compile down to
it can also lead to large bloated applications that can take up many megabytes
of bandwidth, due to Javascript being a text-based format.

Browser vendors like Firefox and Chrome have made countless attempts at
optimizing the performance of JavaScript Recently, a low-level bytecode format
for client-side code called WebAssembly has been standardized. WebAssembly
promises to be faster to both execute and evaluate than ordinary Javascript,
and also take less space. WebAssembly can also be written in various low-level
languages like C, C++ or Rust. Besides WebAssembly, other means of writing
performant client-side code in languages other than JavaScript includes asm.js,
which is a subset of JS that C/C++ can be compiled down to, utilizing some
well-supported performance optimizations of JavaScript that might be cumbersome
to write by hand. The idea behind asm.js is that it’s output can be directly
converted into native code by browsers and their JavaScript runtimes, without
going through an expensive interpretation process. We'll also look at doing
calculations on the GPU with WebGL, since nowadays especially newer mobile
devices contain quite capable GPUs, which are often quite underutilized in web
applications. Naturally, GPUs require a totally different mindset when
designing the actual algorithms since their operation differs fundamentally
from CPUs, rendering them suboptimal for tasks such as graph traversal or
sorting, since they cannot be parallelized easily.

Within this paper we benchmark various different algorithms implemented in
plain JavaScript, the WebGL framework gpu.js, and C/C++ compiled to asm.js and
WebAssembly.  We then benchmark them on various different device and browser
combinations. The algorithms we use are  trivial and well understood, such that
possible subtle differences in implementation won't affect the end result. The
algorithms used are sorting an array, matrix multiplication and finding the nth
Fibonacci number. As is apparent, none of these algorithms are themselves
commonly used in the average web application, but they are used more as a
proof-of-concept for heavy computing on the browser.

\section{Experiment setup}

As was previously stated, three different classic algorithms were chosen for
performaning the benchmarks. They were calculating the nth number in the
Fibonacci sequence, performing multiplication of two square matrices of equal
sizing, and sorting a list of numbers. The matrix multiplication and list
sorting were chosen due to the ease-of-scalability, since we could easily
generate test data for them for arbitrary amounts of size. They also allowed us
to perform benchmarks on different data types, since, for example,
theoretically multiplying two 64-bit floating point numbers is much slower than
doing the same for two 32-bit integers due to the way registers are laid out in
modern CPUs. We theorized this would give WebAssembly a small benefit against
asm.js and JavaScript, since both of them are constrained to using 64-bit
floating point number types for arithmetic calculations, since it is the only
number type available in JavaScript. The Fibonacci sequence was chosen because
we wanted to test the possible differences between a recursive and iterative
version of the algorithm, and we wanted to see how WebAssembly handles
recursion compared to its alternatives, being a stack-machine based system.

For easily running different benchmarks on combinations of various different
devices and browsers, we set up an account on the popular browser testing
service BrowserStack. BrowserStack is originally meant for writing and running
end-to-end test cases for web applications, but for our purposes it fit our
needs perfectly. BrowserStack offers a simple API for programmatically
operating a given browser, and waiting for a certain condition to be fulfilled,
so the core testing code was only 20 lines of JavaScript or so. The testing
code can be in listing~\ref{stacktest}. In it, we simply iterate over the
various browsers, our test cases and their various parameters and launch an
asynchronous BrowserStack session for each of them. BrowserStack were also
extremely generous in offering us their paid services for absolutely no cost,
and we would very much like to extend our thanks for that.

\begin{lstlisting}[caption={BrowserStack testing code},label=stacktest, basicstyle=\scriptsize]
for (const c of capabilitiesUsed) {
  for (const p of projects) {
    for (const params of p.parameters) {
      for (t of p.testsAvailable) {
        console.log(
          `Running test ${p.name} ${t} on browser ${
            c.browserName
          } with query parameters ${params}`
        );

        const url = `${baseUrl}/${p.name}/${t}/?${params}`;

        doRun(url, c);
      }
    }
  }
}
\end{lstlisting}

All in all, we chose to test on nine different browsers. For all nine, we used the latest available version.

\begin{itemize}
  \item Internet Explorer 11
  \item Safari on macOS
  \item Opera
  \item Edge
  \item Safari on iPhone
  \item Safari on iPad
  \item Android default WebView browser
  \item Firefox
  \item Chrome
\end{itemize}

\section{Results}
\section{Analysis}
\section{Conclusions}

\bibliographystyle{IEEEtran}
\bibliography{report}

\end{document}

